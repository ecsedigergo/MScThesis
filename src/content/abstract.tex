% !TeX spellcheck = hu_HU

\pagenumbering{roman}
\setcounter{page}{1}

\selecthungarian

%----------------------------------------------------------------------------
% Abstract in Hungarian
%----------------------------------------------------------------------------
\chapter*{Kivonat}\addcontentsline{toc}{chapter}{Kivonat}

Egy hardver és szoftver elemekből álló összetett alkalmazás esetén a meghibásodások számos komponensből eredhetnek, ráadásul egy ilyen többrétegű rendszerben a hiba okának feltárása és hibás működés elhárítása is komoly feladat. A biztonságkritikus rendszerekben mindenképpen meg kell előzni a végzetes, akár emberi életekbe kerülő hibák előfordulását. Az ilyen komplex rendszerek tesztelését érdemes részletesen megtervezni és dokumentálni a céljainknak megfelelően.

Munkám során a Méréstechnika és Információs Rendszerek Tanszéken folyamatosan fejlesztett modellvasutakra épülő demonstrátort vettem alapul mint többrétegű biztonságkritikus kiberfizikai rendszert. Az eddigi teszt megvalósítások a keretrendszer gyors fejlődése miatt hamar elavultak és nehezen karbantarthatók lettek.  A különböző fejlesztési módszerekhez és eszközökhöz akár teljesen eltérő tesztelési stratégiákat kell alkalmaznunk, hogy megbizonyosodjunk minden réteg hibamentes és biztonságos működéséről. Az ideális tesztelési stratégia kialakításához meg kell ismernünk a tesztelendő rendszer architektúráját és felhasznált technológiákat, hogy meg tudjuk határozni azokat a pontokat ahol érdemes teszteket a rendszerbe illeszteni.
\todo[inline]{include goals and achivements}

\vfill
\selectenglish


%----------------------------------------------------------------------------
% Abstract in English
%----------------------------------------------------------------------------
\chapter*{Abstract}\addcontentsline{toc}{chapter}{Abstract}

In a complex system, where hardware and software  elements met and interact with each other, a system failure can propagate from different components and even from different components. These failures are unacceptable in a safety-critical system, like in a railway system. Through all the development phases we can add verification steps to improve the whole system failure rate.

My thesis work is based on a railway demonstrator system developed by the students of Measurement and Information Systems department, which simulates a safety-critical railway track. Aim of the system is to demonstrate the model driven and verification techniques through the development life cycle. To satisfy this purpose custom made and off-the-shelf products have been added in hardware and software level also. For different development techniques there must be different verification approaches to be sure that our safety-critical system will be stable and highly reliable. The current state of the demonstrator satisfy this approach, therefore I have investigated and designed test approaches for improving the demonstrator reliability in component, integration and system level.


\vfill
\selectthesislanguage

\newcounter{romanPage}
\setcounter{romanPage}{\value{page}}
\stepcounter{romanPage}