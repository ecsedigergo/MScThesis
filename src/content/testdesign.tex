%----------------------------------------------------------------------------
\chapter{Test design}
%----------------------------------------------------------------------------
\section{Test approaches}
The MoDeS$^3$ is a multi-layered application with safety-critical functionalities. As a complex system with many off-the-self and custom made components, it is crucial to have a well-designed testing strategy. For this purpose in the following phases I will examine the possible and most suitable test approaches.

\begin{figure}[h]
	\centering
	\includegraphics[width=150mm]{figures/testDesign/V_model.png}
	\caption{V-model}
	\label{fig:vModel}
\end{figure}

\paragraph{V-model and testing levels}
Nowadays one of the most popular software development methodology is the V-model \cite{Vmodel} (shown on figure \ref{fig:vModel}), which defines four stages during the development and four verification steps accordingly (shown with rectangles). In addition for making the development more effective for error-detection, there are four more steps defined for verifying our system \cite{TestLevels} (shown with ellipses). While the development starts with an abstract \textbf{requirement analysis}, which declares the aim of our application, later on each step contains more detailed information. The second phase is about understanding the abstract user requirements and define the \textbf{system functional specification} by the developers. During this phase our verification is based on \textbf{Model-in-the-Loop (MiL) testing}. The third phase contains all the low-level design and specific information about the application. From these design decisions, the \textbf{implementation} should be quite straight-forward or even generated. To make sure that the code is working correctly it is tested using \textbf{Software-in-the-Loop (SiL)}. 
As of now our application code is written and our design decisions are verified, we make sure with \textbf{unit tests} that our functions are working well on their own. \textbf{Processor-in-the-Loop (PiL)} testing can be made here. Moving up in the V-model with \textbf{integration tests}, we are focusing on more abstract and complex components. In this phase we consider software and hardware elements also, completed with \textbf{Hardware-in-the-Loop (HiL)} testing. The last step is about verifying the complete system in real environment and check the high-level criterias.

\todo[inline]{Verify test cases}

\section{Test scenarios for MoDeS$^3$}
In the following section I will investigate the test opportunities for the MoDeS$^3$ system.  The \ref{table:test_cases} table shows the software components for different test levels and layouts. In the table a check mark shows the possibility to test that component in that exact level/layout. For the other cases testing is irrelevant. I have defined four integration test layouts, and for each test layout a component is involved where a check mark is shown. All the software components, which is deployed to other hardware components have network communication, therefore the messenger software component is also involved in that test.

\subsection{Unit test}
Those components in the system which have some logic or crucial functional responsibility, should have been verified separately also. Furthermore not all the software component have been mentioned on the \ref{table:test_cases} table, because not safety critical or purely hardware components have been skipped. \textit{Barrier} is irrelevant regarding the train-collision detection. How the trains get speed and direction controls is not safety critical either, so \textit{LeapMotion} and \textit{XPressNet} components are not safety critical. \textit{Touchboard} and \textit{Dashboard} components are only visualizing the railway current states send commands to the system, so these are not relevant elements in terms of train collision.

\begin{table}[h]
	\caption{Component test possibilities in test levels}
	\label{table:test_cases}
	\begin{center}
		\renewcommand{\arraystretch}{1.8}
		\begin{tabu} 
			to 0.9 \textwidth
			 {  m{4.5 cm}  c  X[c] X[c] X[c] X[c]  X[c]  }
			\toprule
			\centeredDoubleRow{2}{Component Name} & \centeredDoubleRow{2}{Component} &          \multicolumn{4}{c}{Integration}          & System     \\
			\cmidrule(rl){3-6} \cmidrule(l){7-7}  &                                  & layout 1   & layout 2   & layout 3   & layout 4   & layout 1   \\ \midrule
			Section Occupancy Query               & \checkmark                       & \checkmark &            & \checkmark &            & \checkmark \\
			Occupancy Query                       & \checkmark                       & \checkmark &            & \checkmark &            & \checkmark \\
			GPIO                                  & \checkmark                       &            & \checkmark &            & \checkmark & \checkmark \\
			Track Element Controller              & \checkmark                       &            & \checkmark &            & \checkmark & \checkmark \\
			Safety Logic                          & \checkmark                       &            &            & \checkmark & \checkmark & \checkmark \\
			Messenger                             &                                  & \checkmark &            & \checkmark & \checkmark & \checkmark \\ \bottomrule
		\end{tabu}
	\end{center}
\end{table} 

\todo[inline]{Create one overall diagram for all the layouts}

\subsection{Integration test}
\paragraph{Integration test layout 1}
Aim of this test case is to check the occupancy detection functionality, which was described in \ref{section:OccupancyDetection} section. One track element is occupied if a train is on that element (even moving or stopped). In hardware point of view this detection is made by the DigiSens-8-S88 off-the-shelf sensor product, so it is unnecessary to test it.

On figure \ref{fig:MoDeS3_Deployment-test1} the affected software components are highlighted with yellow marks. We can remain with testing only the software elements in this integration test.
\begin{figure}[!h]
	\centering
	\includegraphics[width=150mm, keepaspectratio]{figures/testDesign/testLayoutSYSML/MoDeS3_Deployment-test1.png}
	\caption{Integration test layout 1}
	\label{fig:MoDeS3_Deployment-test1}
\end{figure}

\paragraph{Integration test layout 2}
Aim of this test is to verify the software components, which is deployed to the \textit{BBB} and responsible for controlling the physical segments (see \ref{par:FunctionTEC} section for more description about this function). All the 6 \textit{BBB} have separate instance of \textbf{Track Element Controller} software function. This component gets an information through the network on command topics (shown in \ref{par:MQTTTopicCommand} section), when segment allowance or turnout state must be changed. At this point the control differentiates into subcomponents weather a turnout or a section command have been received. After command execution the \textit{Track Element Controller} sends the new status on the \textit{status topic} (see \ref{par:MQTTTopicStatus}). Both branches uses GPIO software component for giving impulse on the BBB cape pins.
\begin{figure}[!h]
	\centering
	\includegraphics[width=150mm, keepaspectratio]{figures/testDesign/testLayoutSYSML/MoDeS3_Deployment-test2.png}
	\caption{Integration test layout 2}
	\label{fig:MoDeS3_Deployment-test2}
\end{figure}

\paragraph{Integration test layout 3}
In this integration test we want to verify the decisions of \textit{Safety Logic} component. After we have verified the \textit{Safety Logic} as a unit and the occupancy detection elements, we can investigate the \textit{Safety Logic} decision based on the generated occupancy states together. These signs can be injected into the \textit{Arduino} hardware element or into the Section Occupancy Query directly.
\begin{figure}[!h]
	\centering
	\includegraphics[width=150mm, keepaspectratio]{figures/testDesign/testLayoutSYSML/MoDeS3_Deployment-test3.png}
	\caption{Integration test layout 3}
	\label{fig:MoDeS3_Deployment-test3}
\end{figure}

\paragraph{Integration test layout 4}
The fourth integration test should be a verification of an action after a \textit{Safety Logic} decision. In details, we can inject a critical scenario into the \textit{Safety Logic} component where a safety intervention needed and check the GPIO component's file writing operations.
\begin{figure}[!h]
	\centering
	\includegraphics[width=150mm, keepaspectratio]{figures/testDesign/testLayoutSYSML/MoDeS3_Deployment-test4.png}
	\caption{Integration test layout 4}
	\label{fig:MoDeS3_Deployment-test4}
\end{figure}

\todo[inline]{Extend test cases}

\subsection{System test}
In the system tests all components are involved and only the injected, specific scenario is different. 
I have already mentioned a few collision scenario in \ref{par:trainScenarios} section, and I just want to add three more scenarios with the double-turnout. First of them (see figure \ref{fig:LayoutT3-scenario1}) is a typical case, where the collision can be easily avoided with \textit{Train 1} waiting.
\begin{figure}[!h]
	\centering
	\includegraphics[width=150mm, keepaspectratio]{figures/modes3/layoutT3-scenario1.png}
	\caption{Turnout 3 collision scenario 1}
	\label{fig:LayoutT3-scenario1}
\end{figure}

The next possible collision could happen, when our two trains going exactly into each other after 2 sections.
\begin{figure}[!h]
	\centering
	\includegraphics[width=150mm, keepaspectratio]{figures/modes3/layoutT3-scenario2.png}
	\caption{Turnout 3 collision scenario 2}
	\label{fig:LayoutT3-scenario2}
\end{figure}
\todo[inline]{Only With the bottom track its more understandable}

The third scenario is similar with the first one, but if our T3 turnout is in a right direction (both turnout is in the straight direction), our trains are safe.
\begin{figure}[!h]
	\centering
	\includegraphics[width=150mm, keepaspectratio]{figures/modes3/layoutT3-scenario3.png}
	\caption{Turnout 3 collision scenario 3}
	\label{fig:LayoutT3-scenario3}
\end{figure}