%----------------------------------------------------------------------------
\chapter{\bevezetes}
%----------------------------------------------------------------------------
Nowadays in safety critical systems the usage of more high level software components have increased. Therefore the need for building a system which contains hardware and software parts also have increased. In every safety critical area the most important criteria is that our system and also our components must be reliable on their own. Because of the system's complexity to satisfy these strong requirements is hardly manageable and understandable for the first sight. 


In my thesis, I will introduce a demonstrator railway system which models a safety critical railway track. The Model-based Demonstrator for Smart and Safe Systems (MoDeS$^3$) project is a BME MIT, BME-MTA, IncQuery Labs Kft. and Quanopt Kft. cooperation. For the demonstrator it is also inevitable to verify each component appropriate process and interaction with other elements. At first I introduce the basic hardware parts and layout for model railway systems. Then give details about the extended hardware elements, which makes the demonstrator able to detect and avoid train collision scenarios. For this purpose custom software components have also been implemented. For creating a test design, I have collected the communication and functional dependencies for all components. Finally I have designed a specific test approach for the MoDeS$^3$ system.

